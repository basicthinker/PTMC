
\section{User Manual}

The following should eventially become a real manual. For now, you'll find
more detailed descriptions of some of the issues here.

\subsection{Indirect Calls / Function Pointers}

When a function is to be called during a transaction, the compiler must make
sure that the code that will be executed is the transactional version of this
function. If the call is indirect (i.e., the call target is not fixed in
the program code), the compiler has to make sure that the input function
pointer to nontransactional code is converted into a pointer to the
respective transactional version of the function.

There are basically two ways to perform this mapping: (1) getting the
transaction version directly from somwhere in the binary code of the original
function and (2) looking it up in some data structure. The former is
difficult because Tanger operates on the LLVM intermediate representation of
the program and we do not want to change backends. The latter is easy to
implement but has higher runtime costs.

For now, only the latter option is implemented. It uses the lookup functions
implemented in Tanger's \texttt{stmsupport} library. This means that you have
to build link this library if you want to use indirect calls. If a there is
an indirect call that cannot be resolved at runtime, the stmsupport library
will abort the execution of the program.

To use indirect calls, you must make sure that Tanger knows for which functions
it has to create a mapping from the nontransactional to the transactional
version. First, you can mark respective functions explicitely by putting a call
to the \texttt{tanger_indirect_call_target()} function into the respective
function (the call will be detected and removed by Tanger). Second, you can
let Tanger assume that all functions whose address is taken somewhere in the
program are potential targets of transactional indirect calls. If
you compile your complete program using Tanger, then there is no
need for any annotations in the source code of your program but the number of
pairs can potentially be larger and thus increase the lookup overhead.
Currently, the second option is enabled by default.


